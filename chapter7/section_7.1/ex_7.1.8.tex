% === Exercise 7.1.8 ===
\begin{Exercise}
\begin{proof}
Consider $$f_n(x) = \left( 1+\frac{x}{n} \right)^n e^{-x}\text{ and } f(x)=\lim_{n\to\infty}f_n(x) = 1.$$
Since each $f_n$ is continuous on $\mathbb{R}$, then so is integrable.
For any $x\in\mathbb{R}$, we have $$ \left| f_n(x) - f(x) \right| =
\left| e^{-x} \left( \left( 1+\frac{x}{n} \right)^n - e^x \right) \right| \leq
\left| \left( 1+\frac{x}{n} \right)^n - e^x \right|
\to0 \mbox{ as } n\to\infty.$$
Given $\epsilon > 0$, $\exists N\in\mathbb{N}$ such that $$n \geq N \implies \left| f_n(x) - f(x) \right| < \epsilon,$$
so $f_n \to f$ uniformly on $\mathbb{R}$ implies $f$ is integrable on $\mathbb{R}$. Finally, we conclude $$\lim_{n\to\infty} \int_{a}^{b} \left( 1+\frac{x}{n} \right)^n e^{-x}\ dx =
\int_{a}^{b} \lim_{n\to\infty} \left( \left( 1+\frac{x}{n} \right)^n e^{-x}\right)\ dx =
\int_{a}^{b} dx =
b-a.$$
\end{proof}
\end{Exercise}