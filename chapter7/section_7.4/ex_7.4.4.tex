% === Exercise 7.4.4 ===
\begin{Exercise}
\begin{proof}
By hypothesis, we suppose $$P(x) = f(x) = a_0+a_1 x + \cdots + a_n x^n.$$
Since $P\in \mathbb{C}^{\infty}(-\infty, \infty)$, by uniqueness, we set $$\beta_k = \begin{cases} \frac{f^{(k)}(x_0)}{k!} & \mbox{ for } k=0, 1, \cdots, n \\
0 & \mbox{ for } k \geq n+1 \end{cases}.$$
Hence, the Taylor expansion of $P$ centered at $x_0$ is $$P(x) = \sum_{k=0}^{\infty}\frac{f^{(k)}(x_0)}{k!} (x-x_0)^k =
\sum_{k=0}^{\infty}\beta_k (x-x_0)^k =
\beta_0 + \beta_1(x-x_0)+\cdots+\beta_n(x-x_0)^n$$ for all $x\in\mathbb{R}$ as promised.
\end{proof}
\end{Exercise}