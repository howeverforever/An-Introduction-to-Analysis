\documentclass{report}
\usepackage[utf8]{inputenc}
\usepackage[a4paper, total={6in, 8in}]{geometry}
\usepackage{amsmath}
\usepackage{amsthm}
\usepackage{fancyhdr}
\usepackage{natbib}
\usepackage{mathpazo}
\usepackage{graphicx}
\usepackage{enumerate}
\usepackage{type1cm}
\usepackage{exercise,chngcntr}

\newenvironment{solution}
  {\begin{proof}[Solution]}
  {\end{proof}}

\pagestyle{fancy}
\fancyhf{}
\rhead{\rightmark}
\rfoot{\thepage}

\begin{document}

\counterwithin{Exercise}{section}

\setcounter{chapter}{5}
\chapter{Infinite Series of Real Numbers}
\thispagestyle{empty}
\newpage

\section{Introduction}
\setcounter{Exercise}{3}
% === Exercise 6.1.4 ===
\begin{Exercise}
\begin{solution}
\begin{flalign*}
&\sum_{k=1}^{\infty} (a_{k+1}-2a_k+a_{k-1}) = \lim_{n\to\infty} (a_0-a_1+a_{n+1}-a_n) = a_0-a_1 &
\end{flalign*}
\end{solution}
\end{Exercise}

\vspace{12pt}
% === Exercise 6.1.5 ===
\begin{Exercise}
\begin{solution}
\begin{flalign*}
 &\sum_{k=1}^{\infty}(x^k-x^{k-1})(x^k+x^{k-1}) = \sum_{k=1}^{\infty}(x^{2k}-x^{2k-2}) = \lim_{n\to\infty}(x^{2n}-1)\ \text{converges} &\\
\iff& |x| \leq 1 &
\end{flalign*}
If $x = \pm 1$, then the value of the series is $0$, \\
otherwise $x \in (-1,1)$, the value of the series is $-1$.
\end{solution}
\end{Exercise}

\setcounter{Exercise}{6}
\vspace{12pt}
% === Exercise 6.1.7 ===
\begin{Exercise}
\begin{enumerate}[a)]
\item 
\begin{proof}
Since $f'(x)$ exists for all $x \in \mathbb{R}$, we conclude $F'(x)$ exists. \\
By the Mean Value Theorem, there is a number $c \in I$ so that $$ |F(x)-F(y)| = F'(c)(x-y),\ \forall x,y \in I$$
where $$F'(c) = 1-\frac{f'(c)}{f'(a)}$$
Since $c \in I$, we know $\frac{f'(c)}{f'(a)} \in [1-r,1]$. Therefore $$ 0 \leq F'(c) = 1-\frac{f'(c)}{f'(a)} \leq r$$
As a result, we conclude that $$|F(x)-F(y)| = |F'(c)||x-y| \leq r|x-y|,\ \forall x,y \in I $$
\end{proof}

\item
\begin{proof}
For $n=1$, by definition of $x_n$, we get $$|x_2-x_1| = |F(x_1)-F(x_0)| \leq r|x_1-x_0|$$
Assume for $n=k$, $$|x_{k+1}-x_{k}| \leq r^k|x_1-x_0|$$ holds, then for $n=k+1$, $$|x_{k+2}-x_{k+1}| = |F(x_{k+1})-F(x_{k})| \leq r|x_{k+1}-x_{k}| \leq r^{k+1}|x_1-x_0|$$ also holds.
By induction, we conclude $$|x_{n+1}-x_{n}| \leq r^n|x_1-x_0|,\ \forall n \in \mathbb{N}$$
\end{proof}

\item 
\begin{proof}
Since $f(I) \subseteq I$ and $x_0 \in I$, we hold $x_n \in I,\ \forall n \in \mathbb{N}$. And $I$ is a closed interval, by Bolzano-Weierstrass Theorem, there is a subsequence $\{x_{n_k}\}$ which converges to a fixed number $b \in I$.

\vspace{1ex}

Moreover, by part b), we get $ \{x_n\}$  is Cauchy. Hence $$\lim_{n\to\infty} x_n = b $$

Besides, $F$ is differentiable on $I$, so $F$ is continuous on $I$. \\
Consider the equation $x_n = F(x_{n-1})$ and notice $F$ is continuous on $I$. Take the limit on both sides as $n\to\infty$, we have $b=F(b)$ which implies that $$b=b-\frac{f(b)}{f'(a)}$$
Then, we conclude $f(b)=0$, as promised.
\end{proof}
\end{enumerate}
\end{Exercise}


\setcounter{Exercise}{8}
\vspace{12pt}
% === Exercise 6.1.9 ===
\begin{Exercise}
\begin{enumerate}[a)]
\item
\begin{proof}
For all $n>N$,
\begin{flalign*}
&\left|nb-\sum_{k=1}^{n}b_k\right| = \left|\sum_{k=1}^{n}(b-b_k)\right| \leq \sum_{k=1}^{n}|b-b_k| \leq \sum_{k=1}^{N} |b-b_k| + \sum_{k=N+1}^{n}M = \sum_{k=1}^{N}|b_k-b| + M(n-N) &
\end{flalign*}
\end{proof}

\item
\begin{proof}
Since $\lim_{n\to\infty} b_n = b$, given $\epsilon>0$, $\exists N\in\mathbb{N}$ such that $n \geq N \implies |b_n-b| < \epsilon$.
Then if $n > N$, by part a), $$\left|\frac{b_1+b_2+...+b_n}{n}-b\right| = \left|\frac{b_1+b_2+...+b_n-nb}{n}\right| \leq \left|\frac{\sum_{k=1}^{n}\left|b_k-b\right|}{n}\right|+\epsilon(1-\frac{N}{n}) $$
Take the limit of both sides as $n\to\infty$, we have $$\limsup_{n\to\infty}\left|\frac{b_1+b_2+...+b_n}{n}-b\right| \leq \epsilon$$
This hold for any $\epsilon > 0$, so $$\lim_{n\to\infty}\left|\frac{b_1+b_2+...+b_n}{n}-b\right| = 0 $$
which means $$\frac{b_1+b_2+...+b_n}{n}\to b \text{ as } n\to\infty$$
\end{proof}

\item
\begin{solution}
The counter-example is $b_k=(-1)^{k-1}$.
\end{solution}
\end{enumerate}
\end{Exercise}


\setcounter{Exercise}{11}
\vspace{12pt}
% === Exercise 6.1.12 ===
\begin{Exercise}
\begin{proof}
Let $$b_n := \sum_{k=1}^{n} k a_k = \frac{n+1}{n+2} $$
So $$b_{n+1}-b_n=(n+1)a_{n+1}=\frac{1}{(n+2)(n+3)} $$
We have $$a_1=\frac{2}{3}\text{ and } a_n=\frac{1}{n(n+1)(n+2)}\text{ for } n > 1$$
Calculate the summation and we get $$\sum_{k=1}^{\infty} a_k = a_1+\sum_{k=2}^{\infty} a_k = \frac{2}{3}+\frac{1}{12} = \frac{3}{4}$$
\end{proof}
\end{Exercise}


\section{Series with Nonnegative Terms}

% === Exercise 6.2.1 ===
\begin{Exercise}
\begin{enumerate}[a)]
\item 
\begin{proof}
Let $$a_n := \frac{2n+5}{3n^3+2n-1}\text{ and } b_n :=\frac{1}{n^2}$$
Since $\sum_{k=1}^{\infty}\frac{1}{k^2}$ converges by the p-Series Test, besides
$$0<\lim_{n\to\infty}\frac{a_n}{b_n} = \frac{2}{3} < \infty$$
By the Limit Comparison Test, we conclude $$\sum_{k=1}^{\infty}\frac{2k+5}{3k^3+2k-1}$$ converges.
\end{proof}

\item [d)]
\begin{proof}
Since $\log{k} < k,\ \forall k \in \mathbb{N}$, we have $$\sum_{k=1}^{\infty}\frac{k^3\log^2{k}}{e^k} \leq \sum_{k=1}^{\infty}\frac{k^5}{e^k}$$
Let $$a_n := \frac{n^5}{e^n}\text{ and } b_n :=\frac{1}{n^2}$$
Since $\sum_{k=1}^{\infty}\frac{1}{k^2}$ converges by the p-Series Test, besides
$$\lim_{n\to\infty}\frac{a_n}{b_n} = \lim_{n\to\infty}\frac{n^7}{e^n} = 0 $$
By the Limit Comparison Test and the Comparison Test, we conclude $$\sum_{k=1}^{\infty}\frac{k^3\log^2{k}}{e^k}$$ converges.
\end{proof}
\end{enumerate}
\end{Exercise}

\vspace{12pt}
% === Exercise 6.2.2 ===
\begin{Exercise}
\begin{enumerate}[a)]
\item 
\begin{proof}
Let $$a_n := \frac{3n^3+n-4}{5n^4-n^2+1}\text{ and } b_n :=\frac{1}{n}$$
Since $\sum_{k=1}^{\infty}\frac{1}{k}$ diverges by the p-Series Test, besides
$$0<\lim_{n\to\infty}\frac{a_n}{b_n} = \frac{3}{5} < \infty$$
By the Limit Comparison Test, we conclude $$\sum_{k=1}^{\infty}\frac{3k^3+k-4}{5k^4-k^2+1}$$ diverges.
\end{proof}

\item [d)]
\begin{proof}
Consider $$\int_{2}^{\infty}\frac{1}{x\log^p{x}} = \left|(\log{x})^{1-p}\right|_{2}^{\infty} + p\int_{2}^{\infty}\frac{1}{x\log^p{x}}$$
Since $p \leq 1$, then $1-p \geq 0$, we have $$\int_{2}^{\infty}\frac{1}{x\log^p{x}} = \frac{\left|(\log{x})^{1-p}\right|_{2}^{\infty}}{1-p} = \infty$$
By the Integral Test, we conclude $$\int_{2}^{\infty}\frac{1}{x\log^p{x}}\quad for\ p \leq 1$$
diverges.
\end{proof}
\end{enumerate}
\end{Exercise}


\vspace{12pt}
% === Exercise 6.2.3 ===
\begin{Exercise}
\begin{proof}
Assume $a_k \leq M,\ \forall k \in \mathbb{N} $ where $M > 0$. Therefore $$ \frac{a_k}{(k+1)^p} < \frac{M}{k^p} $$
Since $p>1$, by the p-Series Test and the Limit Comparison Test, we know $$\sum_{k=1}^{\infty}\frac{M}{k^p}$$ converges. Then by the Comparison Test, we conclude $$\sum_{k=1}^{\infty}\frac{a_k}{(k+1)^p}$$ converges.
\end{proof}
\end{Exercise}

\setcounter{Exercise}{4}
\vspace{12pt}
% === Exercise 6.2.5 ===
\begin{Exercise}
\begin{proof}
Since for $p \geq 0$, we have $$\frac{|a_k|}{k^p} \leq |a_k|$$
By the Comparison Test, we know $$\sum_{k=1}^{\infty}\frac{|a_k|}{k^p}$$
converges. When $p < 0$, the series might converge or diverge. e.g. set $p=-1$ and $a_k=\frac{1}{k}$, then $$\sum_{k=1}^{\infty}\frac{|a_k|}{k^p} = \sum_{k=1}^{\infty}1$$ which diverges. For the same $p$, set $a_k = \frac{1}{k^3}$, then the series $$\sum_{k=1}^{\infty}\frac{|a_k|}{k^p} = \sum_{k=1}^{\infty}\frac{1}{k^2}$$ converges by the p-Series Test.
\end{proof}
\end{Exercise}

\setcounter{Exercise}{6}
\vspace{12pt}
% === Exercise 6.2.7 ===
\begin{Exercise}
\begin{proof}
Since the series $\sum_{k=1}^{\infty}a_k$ converges, then $a_k \leq M$ for all $k$. It follows that $0 \leq a_kb_k \leq Mb_k$ and the series  $\sum_{k=1}^{\infty}b_k$ converges. By the Comparison Test, we conclude $\sum_{k=1}^{\infty}a_kb_k$ converges.
\end{proof}
\end{Exercise}

\setcounter{Exercise}{8}
\vspace{12pt}
% === Exercise 6.2.9 ===
\begin{Exercise}
\begin{proof}
$(\Longrightarrow)$
Let $s_n = \sum_{k=1}^{\infty}a_k$. Since $$\sum_{k=1}^{\infty}a_k = \lim_{n\to\infty}\sum_{k=1}^{n}a_k = \lim_{n\to\infty}s_n = S \in \mathbb{R}$$
converges, so does its partial sum. Then $$\sum_{k=1}^{\infty}(a_{2k}+a_{2k+1}) = \lim_{n\to\infty}(s_{2n+1}-a_1) = S-s_1 \in \mathbb{R}$$ converges. 

\vspace{2ex}

$(\Longleftarrow)$ Since  $$\sum_{k=1}^{\infty}(a_{2k}+a_{2k+1}) = \lim_{n\to\infty}(s_{2n+1}-a_1)$$ converges so does $s_{2n+1}$. Consider $$s_{2n+2} = s_{2n+1}+a_{2n+2}$$
Taking the limit of both sides as $n\to\infty$, we have $$\lim_{n\to\infty}s_{2n+2} = \lim_{n\to\infty}s_{2n+1}$$
which means $s_{2n+2}$ also converges.

\vspace{2ex}

Combining above discussion, no matter the number of terms is either odd or even, the series always converges. We conclude $$ \sum_{k=1}^{\infty}a_k$$ converges.
\end{proof}
\end{Exercise}
\end{document}