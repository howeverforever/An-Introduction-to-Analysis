% === Exercise 8.3.6 ===
\begin{Exercise}
\begin{enumerate}[a)]
\item
\begin{proof}
$(\Longrightarrow)$
Since $C$ is relatively closed in $E$, then by definition, there exists a closed set $B$ such that $C=E\cap B$. Because $E$ and $B$ are closed, we conclude $C$ is closed.\\

$(\Longleftarrow)$
Since $C\subseteq E$ implies $C=C\cap E$, and we know $C$ and $E$ are closed by hypotheses. By definition, we conclude $C$ is relatively closed in $E$.
\end{proof}

\item
\begin{proof}
$(\Longrightarrow)$
Since $C$ is relatively closed in $E$, there exists a closed set $B$ such that $C=E\cap B$. Consider
$$
C\backslash E 
= C\cap E^c 
= E\cap \left(E\cap B \right)^c 
= E\cap \left( E^c\cup B^c \right) 
= \left( E\cap E^c\right) \cup \left( E \cap B^c\right)
= E\cap B^c.
$$
Since $B^c$ is open, we conclude $E\backslash C$ is relatively open in $E$.\\

$(\Longleftarrow)$
Since $E\backslash C$ is relatively open in $E$, there exists an open set $A$ such that $E\backslash C = E\cap A$. Consider
\begin{align*}
C
&= E\cap C
= \left( E\cap E^c \right) \cup \left(E \cap C\right)
= E\cap \left( E^c \cup C \right) \\
&= E\cap \left( E\cap C^c \right)^c
= E\cap \left( E\cap A \right)^c
= E\cap \left( E^c \cup A^c \right) \\
&= E \cap A^c.
\end{align*}
Since $A^c$ is closed, we conclude $C$ is relatively closed in $E$.
\end{proof}

\end{enumerate}
\end{Exercise}