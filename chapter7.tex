\documentclass{report}
\usepackage[utf8]{inputenc}
\usepackage[a4paper, total={6in, 8in}]{geometry}
\usepackage{amsmath}
\usepackage{amsthm}
\usepackage{fancyhdr}
\usepackage{natbib}
\usepackage{mathpazo}
\usepackage{graphicx}
\usepackage{enumerate}
\usepackage{type1cm}
\usepackage{exercise,chngcntr}

\newenvironment{solution}
  {\begin{proof}[Solution]}
  {\end{proof}}

\pagestyle{fancy}
\fancyhf{}
\rhead{\rightmark}
\rfoot{\thepage}

\begin{document}

\counterwithin{Exercise}{section}

\setcounter{chapter}{6}
\chapter{Infinite Series of Functions}
\thispagestyle{empty}
\newpage

\section{Uniform Convergence of Sequences}
\setcounter{Exercise}{1}
% === Exercise 7.1.2 ===
\begin{Exercise}
\begin{enumerate}[a)]
\item 
\begin{proof}
Consider $$f_n(x) = \frac{n x^{99}+5}{x^3+n x^{66}}\text{ and } f(x)=\lim_{n\to\infty}f_n(x) = x^{33}.$$
Since $f_n$ is continuous for each $n$ on $[1,3]$, then so does integrable.
For any $x\in[1, 3]$, we have $$ \left| f_n(x) - f(x) \right| =
\left| \frac{5-x^{36}}{x^3+n x^{66}} \right| \leq
\frac{5+3^{36}}{x^3+n x^{66}} \leq
\frac{5+3^{36}}{n}\to0 \mbox{ as } n\to\infty.$$
Given $\epsilon > 0$, $\exists N\in\mathbb{N}$ such that $$n \geq N \implies \left| f_n(x) - f(x) \right| < \epsilon,$$
so $f_n \to f$ uniformly on $[1,3]$ implies $f$ is integrable on $[1,3]$. Finally, we conclude $$\lim_{n\to\infty}\int_{1}^{3} \frac{n x^{99}+5}{x^3+n x^{66}}\ dx =
\int_{1}^{3}\left( \lim_{n\to\infty}\frac{n x^{99}+5}{x^3+n x^{66}}\right)dx =
\int_{1}^{3}x^{33} dx =
\left. \frac{x^{34}}{34} \right|_{1}^{3} =
\frac{3^{34}-1}{34}.$$ 
\end{proof}

\item
\begin{proof}
Consider $$f_n(x) = e^{\frac{x^2}{n}}\text{ and } f(x)=\lim_{n\to\infty}f_n(x) = 1.$$
Since $f_n$ is continuous for each $n$ on $[0,2]$, then so does integrable.
For any $x\in[0, 2]$, we have $$ \left| f_n(x) - f(x) \right| =
\left| e^{\frac{x^2}{n}}-1 \right| \leq
e^{\frac{4}{n}}-1\to 0\mbox{ as } n\to\infty.$$
Given $\epsilon > 0$, $\exists N\in\mathbb{N}$ such that $$n \geq N \implies \left| f_n(x) - f(x) \right| < \epsilon,$$
so $f_n \to f$ uniformly on $[0,2]$ implies $f$ is integrable on $[0,2]$. Finally, we conclude $$\lim_{n\to\infty}\int_{0}^{2} e^{\frac{x^2}{n}}\ dx =
\int_{0}^{2} \lim_{n\to\infty}\left( e^{\frac{x^2}{n} }\right) dx =
\int_{0}^{2} dx =
\left. x \right|_{0}^{2} =
2.$$
\end{proof}

\item
\begin{proof}
Consider $$f_n(x) = \sqrt{\sin{\frac{x}{n}}+x+1}\text{, and } f(x)=\lim_{n\to\infty}f_n(x) = \sqrt{x+1}.$$
Since $f_n$ is continuous for each $n$ on $[0,3]$, then so does integrable.
For any $x\in[0, 3]$, we have $$ \left| f_n(x) - f(x) \right| =
\frac{\left| \sin{ \frac{x}{n} } \right| }{  \sqrt{ \sin{ \frac{x}{n} }+x+1} + \sqrt{x+1}} \leq
\sin{\frac{3}{n}}\to 0 \mbox{ as }n\to\infty.$$
Given $\epsilon > 0$, $\exists N\in\mathbb{N}$ such that $$n \geq N \implies \left| f_n(x) - f(x) \right| < \epsilon,$$
so $f_n \to f$ uniformly on $[0,3]$ implies $f$ is integrable on $[0,3]$. Finally, we conclude
\begin{align*}
\lim_{n\to\infty}\int_{0}^{3} \sqrt{\sin{\frac{x}{n}}+x+1}\ dx
&= \int_{0}^{3} \lim_{n\to\infty}\left( \sqrt{\sin{\frac{x}{n}}+x+1} \right) dx \\
&= \int_{0}^{3} \sqrt{x+1}\ dx \\
&= \left. \frac{2}{3} \left( x+1 \right)^{ \frac{3}{2} } \right|_{0}^{3} \\
&= \frac{14}{3}.
\end{align*}
\end{proof}
\end{enumerate}
\end{Exercise}

\vspace{12pt}
% === Exercise 7.1.3 ===
\begin{Exercise}
\begin{proof}
Since $f_n\to f$ uniformly on $E$, pick $\epsilon = 1$ and choose $N\in\mathbb{N}$ such that $$ \left| f_N(x)-f(x) \right|<1\text{ and } \left| f_n(x)-f_N(x) \right|<1 \mbox{ for all } x\in E\text{ and } n\geq N.$$
Since each $f_n$ is bounded, $\exists M_n>0$ such that $$\left|f_n(x)\right|\leq M_n \mbox{ for all } x\in E.$$
Therefore, $$\left| f(x) \right| \leq \left| f_N(x) \right| +1\leq M_n+1\text{ and } \left|f_n(x)\right| \leq \left|f_N(x)\right|+1\leq M_n+1.$$
Set $M := \max\{M_1, M_2, ..., M_N\}+1$, we have $$\left| f_n(x) \right| \leq M\mbox{ for all } x\in E\text{ and } n\in\mathbb{N}$$
which implies ${f_n}$ is uniformly bounded on $E$ and $f$ is a bounded function on $E$.
\end{proof}
\end{Exercise}

\vspace{12pt}
\setcounter{Exercise}{5}
% === Exercise 7.1.6 ===
\begin{Exercise}
\begin{proof}
Since $f_n \to f$ uniformly on $E$, given $\epsilon > 0$, $\exists N\in\mathbb{N}$ such that $$n\geq N \implies \left| f_n(x)-f(x) \right| < \frac{\epsilon}{3}\mbox{ for all } x\in E$$
Consider $f_N$ and since each $f_n$ is uniformly continuous on $E$, for the same $\epsilon$, $\exists \delta>0$ such that $$\left| x-y \right| < \delta \implies \left| f_N(x)-f_N(y) \right| < \frac{\epsilon}{3}\mbox{ for all }x,y\in E.$$
Hence, for any $x, y\in E$ with $\left| x-y \right| < \delta$, we have
\begin{align*}
\left| f(x) - f(y) \right| 
&\leq \left| f(x) - f_N(x) \right| + \left| f_N(x) - f_N(y) \right| + \left| f_N(y) - f(y) \right| \\
&< \frac{\epsilon}{3} + \frac{\epsilon}{3} + \frac{\epsilon}{3} = \epsilon.
\end{align*}
So, $f$ is uniformly continuous on $E$.
\end{proof}
\end{Exercise}


\vspace{12pt}
\setcounter{Exercise}{7}
% === Exercise 7.1.8 ===
\begin{Exercise}
\begin{proof}
Consider $$f_n(x) = \left( 1+\frac{x}{n} \right)^n e^{-x}\text{ and } f(x)=\lim_{n\to\infty}f_n(x) = 1.$$
Since each $f_n$ is continuous on $\mathbb{R}$, then so does integrable.
For any $x\in\mathbb{R}$, we have $$ \left| f_n(x) - f(x) \right| =
\left| e^{-x} \left( \left( 1+\frac{x}{n} \right)^n - e^x \right) \right| \leq
\left| \left( 1+\frac{x}{n} \right)^n - e^x \right|
\to0 \mbox{ as } n\to\infty.$$
Given $\epsilon > 0$, $\exists N\in\mathbb{N}$ such that $$n \geq N \implies \left| f_n(x) - f(x) \right| < \epsilon,$$
so $f_n \to f$ uniformly on $\mathbb{R}$ implies $f$ is integrable on $\mathbb{R}$. Finally, we conclude $$\lim_{n\to\infty} \int_{a}^{b} \left( 1+\frac{x}{n} \right)^n e^{-x}\ dx =
\int_{a}^{b} \lim_{n\to\infty} \left( \left( 1+\frac{x}{n} \right)^n e^{-x}\right)\ dx =
\int_{a}^{b} dx =
b-a.$$
\end{proof}
\end{Exercise}

\end{document}