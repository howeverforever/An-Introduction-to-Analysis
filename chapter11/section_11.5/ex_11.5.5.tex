% === Exercise 11.5.5 ===
\begin{Exercise}
\begin{proof}
When $p=1$, we know
$$
f(x)-f(a) = \int_{0}^{1}D^{(1)} f(a+t h;h) d t.
$$
Since $f$ is $C^p$ on $V$ and $L(x;a) \subset V$, by the Mean Value Theorem, there is $c\in L(x;a)$ such that
$$
\int_{0}^{1}D^{(1)} f(a+t h;h) d t = D^{(1)} f(c;h).
$$
Moreover, applying Taylor's Formula, we have
$$
f(x)-f(a) = D^{(1)} f(c;h).
$$
Hence, the formula holds for $p=1$.

Suppose when $p=q$, the formula holds. We set 
$$
u=D^{(q)} f(a+t h; h) \text{ and } d v = (1-t)^{p-1}d t,
$$
then 
$$
d u=D^{(p+1)} f(a+t h; h)d t\text{ and } v=-\frac{1}{p}(1-t)^p.
$$

By Integration by Part, we have
\begin{align*}
\frac{1}{(q-1)!}\int_{0}^{1}(1-t)^{q-1}D^{(q)}f(a+t h;h)d t
&= \left. -\frac{1}{p}(1-t)^q D^{(q)} f(a+t h;h) \right|_{0}^{1} \\
&\quad + \int_{0}^{1}\frac{1}{q}(1-t)^{q} D^{(q+1)}f(a+t h;h)d t \\
&= \frac{1}{q!}D^{(q)}f(a;h) + \frac{1}{q!}\int_{0}^{1}(1-t)^{q} D^{(q+1)}f(a+t h;h)d t.
\end{align*}
i.e,
\begin{align*}
f(x)-f(a) &
= \sum_{k=1}^{q-1}\frac{1}{k!}D^{(k)}f(a;h)+\frac{1}{(q-1)!}\int_{0}^{1}(1-t)^{q-1}D^{(q)}f(a+t h;h)d t \\
&= \sum_{k=1}^{q}\frac{1}{k!}D^{(k)}f(a;h)+\frac{1}{q!}\int_{0}^{1}(1-t)^{q}D^{(q+1)}f(a+t h;h)d t.
\end{align*}
Hence, we know the formula also holds for $p=q+1$.

By induction, we conclude the formula holds for all $p\in\mathbb{N}$.
\end{proof}
\end{Exercise}