% === Exercise 11.4.11 ===
\begin{Exercise}
\begin{enumerate}[a)]
\item
\begin{proof}
Notice that
$$
D_u f(a) := \lim_{t\to 0}\frac{f(a+t u)-f(a)}{t}
$$
where $\| u \| = 1$.

For each $u\in\mathbb{R}^n$ with $\| u \| = 1$, let $g(t) = f(a+t u)$, then differentiate both sides in $t$, we have $\nabla g(t) = \nabla f(a+t u)\cdot u$.

Then we consider
\begin{align*}
D_u f(a) 
&= \lim_{t\to 0}\frac{f(a+t u)-f(a)}{t} \\
&= \lim_{t\to 0}\frac{g(t)-g(0)}{t} \\
&= \nabla g(0) \\
&= \nabla f(a)\cdot u.
\end{align*}
\end{proof}

\item
\begin{proof}
By hypotheses, we have
$$
\nabla f(a) \cdot u = \| \nabla f(a) \| \| u \| \cos \theta.
$$
From part a) and $\| u \| = 1$, we conclude
$$
D_u f(a) = \nabla f(a)\cdot u = \| \nabla f(a) \| \| u \| \cos \theta =  \| \nabla f(a) \| \cos \theta.
$$
\end{proof}

\item
\begin{solution}
From part b), we know $\theta \in [0,\frac{\pi}{2}]$ which implies 
$$
0\leq D_u f(a) \leq \| \nabla f(a) \|.
$$

The maximum of $D_u f(a)$ is $\| \nabla f(a) \|$ if and only if $\theta = 0$. This occurs when $u$ is parallel to $\nabla f(a)$.
\end{solution}
\end{enumerate}
\end{Exercise}