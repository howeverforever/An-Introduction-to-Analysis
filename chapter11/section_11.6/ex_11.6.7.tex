% === Exercise 11.6.7 ===
\begin{Exercise}
\begin{proof}
By hypothesis, we use the Implicit Function Theorem. Since $F(a) = 0$, and furthermore, $F_{x_j}(a) \neq 0$ implies
$$
\frac{\partial(F_1,\cdots,F_n)}{\partial(x_1,\cdots,x_n)}(a) \neq 0,
$$
then there exists open sets $W_j$ containing $(a_1,\cdots,a_{j-1},a_{j+1},\cdots,a_n)$, and a unique continuously differentiable function $g_j(u^{(j)})$ such that $F(x_1,\cdots,x_{j-1},g_j(u^{(j)}),x_{j+1},c\dots,x_n)=0$ for all $u^{(j)}\in W^j$.

Notice that $g_j(u^{(j)}$ is continuously differentiable on $W^j$, then $g_j(u^{(j)}$ is $C^1$ on $W^j$.

For any $j=1,\cdots, n$, we know the open set $W^j$ contains $(a_1,\cdots,a_{j-1},a_{j+1},\cdots,a_n)$, so there is $r_k^j$ with $1\leq k\leq n$ and $k\neq j$ such that the Cartesian product $(a_1-r_1^j,a_1+r_1^j)\times\cdots\times(a_{j-1}-r_{j-1}^j,a_{j-1}+r_{j-1}^j)\times(a_{j+1}-r{j+1}^j,a_{j+1}+r_{j+1}^j)\times\cdots\times(a_n-r_n^j,a_n+r_n^j)
$ is a subset of $W^j$. Hence for $a_j$, we pick
$$
r_j = \min_{1\leq k\leq n, k\neq j}\{r_k^j\},
$$
then the Cartesian product $(a_1-r_1,a_1+r_1)\times\cdots\times(a_{j-1}-r_{j-1},a_{j-1}+r_{j-1})\times(a_{j+1}-r{j+1},a_{j+1}+r_{j+1})\times\cdots\times(a_n-r_n,a_n+r_n)$ contains $a=(a_1,\cdots,a_j,\cdots,a_n)$.

Fix $a_j$, pick $r>0$ with $r=\min_{1\leq j \leq n}\{r_j\}$ such that $(x_1,\cdots,a_j,\cdots,x_n)\in B_r(a)$. Since $F_{x_j}(a) \neq 0$, then $F_{x_j}(x) \neq 0$ for all $x\in B_r(a)$. 

Moreover, for all $x\in B_r(a)$, we consider
$F_j := F(x_1,\cdots,x_{j-1},g_j(u^{(j)}),x_{j+1},c\dots,x_n)=0$ on $W^j$, then we differentiate both sides on $F_j$ over $x_{j-1}$ for $2\leq j \leq n$, and $F_1$ over $x_n$, we obtain
$$
\frac{\partial F}{\partial x_n} + \frac{\partial F}{\partial x_1}\frac{\partial g_1}{\partial x_n} = 0\text{, and }
\frac{\partial F}{\partial x_j} + \frac{\partial F}{\partial x_{j-1}}\frac{\partial g_j}{\partial x_{j-1}} = 0 \mbox{ for } 2\leq j \leq n
$$
which implies
$$
\frac{\partial g_1}{\partial x_n} = -\frac{\frac{\partial F}{\partial x_n}}{\frac{\partial F}{\partial x_1}}\text{, and }
\frac{\partial g_j}{\partial x_{j-1}} = -\frac{\frac{\partial F}{\partial x_j}}{\frac{\partial F}{\partial x_{j-1}}} \mbox{ for } 2\leq j \leq n.
$$

Finally, by multiplying these $n$ equations, we conclude
$$
\frac{\partial g_1}{\partial x_n}\frac{\partial g_2}{\partial x_1}\frac{\partial g_3}{\partial x_2}\cdots\frac{\partial g_n}{\partial x_{n-1}} = (-1)^n
$$
on $B_r(a)$ as promised.
\end{proof}
\end{Exercise}