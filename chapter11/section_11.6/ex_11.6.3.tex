% === Exercise 11.6.3 ===
\begin{Exercise}
\begin{proof}
Set
\begin{align*}
F_1(u,v,w,x,y) &= u^5+x v^2-y+w; \\
F_2(u,v,w,x,y) &= v^5+y u^2-x+w; \\
F_3(u,v,w,x,y) &= w^4+y^5-x^4-1; \\
F(u,v,w,x,y) &= (F_1,F_2,F_3).
\end{align*}
We know $F(1,1,-1,1,1) = (0,0,0)$. Also,
$$
\frac{\partial(F_1,F_2,F_3)}{\partial(u,v,w)} = \det\begin{vmatrix}
5u^4 & 2x v & 1 \\
2y u & 5v^4 & 1 \\
0 & 0 & 4w^3
\end{vmatrix}
= 4w^3(25u^4v^4-4x y u v)
= -84
\neq 0
$$
when $(u,v,w,x,y)=(1,1,-1,1,1)$.

Hence, by the Implicit Function Theorem, there is $r>0$ such that $B_r(1,1)$ containing $(1,1)$ and a unique continuously differentiable function $g$ such that 
$$
g(x,y) = (u(x,y), v(x,y), w(x,y))\text{ with }g(1,1) = (1,1,-1).
$$
\end{proof}
\end{Exercise}