% === Exercise 11.6.1 ===
\begin{Exercise}
\begin{enumerate}[a)]
\item
\begin{proof}
Consider
$$
f'(t_0) = \begin{bmatrix}
u'(t_0) \\
v'(t_0)
\end{bmatrix} 
\neq
\begin{bmatrix}
0 \\
0
\end{bmatrix}.
$$
Hence, $u'(t_0)$ and $v'(t_0)$ cannot both be zero by definition of a matrix clearly.
\end{proof}

\item
\begin{proof}
Since $f'(t_0)\neq 0$, and $f$ is $C^2$ on $\mathbb{R}$ so is $C^1$ on $\mathbb{R}$. 

Suppose $u'(t_0)\neq 0$, by the Inverse Function Theorem, there is an open set $W$ containing $t_0$ and a unique continuously differentiable 1-1 function $g$ on $f(W)$ containing $x_0$ such that $g(x) = t$ for all $x\in f(W)$. Notice that $g$ is $C^1$ on $f(W)$.

Since $x_0 \in f(W)$, then $g(x_0) = t_0$. For $x$ near $x_0$ which means $x\in f(W)$, we have $g(x) = t$. Hence,
$u(g(x)) = u(t) = x$.

Suppose $v'(t_0)\neq 0$, then a similar argument establishes there is a $C^1$ function $h$ such that $h(y_0)=t_0$ and $v(h(y)) = y$ for $y$ near $y_0$.

From part a), $u'(t_0)$ and $v'(t_0)$ cannot both be zero. Therefore, we conclude either one of two statements will hold.
\end{proof}
\end{enumerate}
\end{Exercise}