% === Exercise 11.2.5 ===
\begin{Exercise}
\begin{proof}
Notice that $\alpha<\frac{3}{2}$, we calculate
$$
f_x(0,0) 
= \lim_{h\to 0} \frac{f(h,0)-f(0,0)}{h} 
= \lim_{h\to 0} h^{3-2\alpha}
= 0.
$$
Similarly, $f_y(0,0) = 0$. We know $D f(0,0) = (0,0)$, then we calculate
$$
\frac{f(h,k)-f(0,0)-D f(0,0)\cdot(h,k)}{\| (h,k) \|}
= \left( h^4+k^4 \right) \left(h^2+k^2 \right)^{\frac{1}{2}-\alpha} \to 0 \text{ as } (h,k)\to(0,0).
$$
Hence, $f$ is differentiable at $(0,0)$.

For $(x,y) = \mathbb{R}^2 \backslash (0,0)$, we calculate
\begin{align*}
f_x(x,y) &= \frac{4x^3(x^2+y^2)-2\alpha x(x^4+y^4)}{(x^2+y^2)^{\alpha+1}}; \\
f_y(x,y) &= \frac{4y^3(x^2+y^2)-2\alpha y(x^4+y^4)}{(x^2+y^2)^{\alpha+1}},
\end{align*}
are continuous on $\mathbb{R}^2 \backslash (0,0)$, and hence $f$ is differentiable on $\mathbb{R}^2 \backslash (0,0)$.

We conclude $f$ is differentiable on $\mathbb{R}^2$ for all $\alpha<\frac{3}{2}$.
\end{proof}
\end{Exercise}