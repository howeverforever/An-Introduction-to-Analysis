% === Exercise 11.2.6 ===
\begin{Exercise}
\begin{proof}
Calculate
$$
f_x(0,0) 
= \lim_{h\to 0} \frac{f(h,0)-f(0,0)}{h} 
= \lim_{h\to 0} \frac{0}{h}
= 0.
$$
Similarly, $f_y(0,0) = 0$. We know $D f(0,0) = (0,0)$, then we calculate
\begin{align*}
\left| \frac{f(h,k)-f(0,0)-D f(0,0)\cdot(h,k)}{\| (h,k) \|} \right|
&= \frac{|h k|^{\alpha} \left| \log(h^2+k^2) \right|}{\sqrt{h^2+k^2}} \\
&\leq \frac{|\frac{h^2+k^2}{2}|^{\alpha} \left| \log(h^2+k^2) \right|}{\sqrt{h^2+k^2}} \\
&= \frac{1}{2^{\alpha}} \frac{|h^2+k^2|^{\alpha}\left| \log(h^2+k^2) \right|}{\sqrt{h^2+k^2}} \\
&\leq \frac{1}{2^{\alpha}} \frac{|h^2+k^2|^{\alpha}\left| (h^2+k^2) \right|}{\sqrt{h^2+k^2}} \\
&= \frac{1}{2^{\alpha}} (h^2+k^2)^{\alpha+\frac{1}{2}} \to 0 \text{ as } (h,k)\to(0,0).
\end{align*}
where $\alpha>\frac{1}{2}$. Hence, $f$ is differentiable at $(0,0)$. 
\end{proof}
\end{Exercise}