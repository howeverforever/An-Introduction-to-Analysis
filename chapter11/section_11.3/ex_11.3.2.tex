% === Exercise 11.3.2 ===
\begin{Exercise}
\begin{enumerate}[a)]
\item
\begin{solution}
Consider
$$
\nabla f(x,y) = (2x,2y),
$$
then
$$
\nabla f(1,-1) = (2,-2).
$$
Hence,
$$
n = (2,-2,-1).
$$
We conclude the tangent plane $\Pi_n(c)$ is
$$
2(x-1)-2(y-1)-z = 2
$$
which implies
$$
2x-2y-z=2.
$$
\end{solution}

\item
\begin{solution}
Consider
$$
\nabla f(x,y) = (3x^2 y-y^3,x^3-3x y^2),
$$
then
$$
\nabla f(1,1) = (2,-2).
$$
Hence,
$$
n = (2,-2,-1).
$$
We conclude the tangent plane $\Pi_n(c)$ is
$$
2(x-1)-2(y-1)-z = 0
$$
which implies
$$
2x-2y-z=0.
$$
\end{solution}

\item
\begin{solution}
Consider
$$
\nabla f(x,y,z) = (y,x,\cos z),
$$
then
$$
\nabla f(1,0,\frac{\pi}{2}) = (0,1,0).
$$
Hence,
$$
n = (0,1,0,-1).
$$
We conclude the tangent plane $\Pi_n(c)$ is
$$
y-w=-1.
$$
\end{solution}
\end{enumerate}
\end{Exercise}