% === Exercise 10.4.8 ===
\begin{Exercise}
\begin{enumerate}[a)]
\item
\begin{proof}
W.L.O.G., we assume
$$
H_1 \subset H_2 \subset \cdots.
$$

Suppose to contrary that
\begin{equation}\label{eq:cantor_counter}
\bigcap_{k=1}^{\infty}H_k = \phi.
\end{equation}
For each $k\in\mathbb{N}$, we set $V_k = X\backslash H_k$. Taking the complement of both sides on \eqref{eq:cantor_counter}, we have
$$
X = \bigcup_{k=1}^{\infty}V_k.
$$
Then $\{V_k\}$ is an open covering of $X$. Since $X$ is compact, we extract a finite subcover, say $V_1 \supset V_2 \supset \cdots$ and $V_i = X$ for some $i\in\mathbb{N}$. It follows that $H_i = X\backslash V_i = \phi$ which is a contradiction to $H_i$ is nonempty by hypothesis.

Hence, we conclude
$$
\bigcap_{k=1}^{\infty}H_k \neq \phi.
$$
\end{proof}

\item
\begin{proof}
\begin{itemize}
\item $\mathbf{(\sqrt{2}, \sqrt{3})\cap \mathbb{Q}\  is\ closed\ and\ bounded}$

$(\sqrt{2}, \sqrt{3})\cap \mathbb{Q}$ is bounded trivially. Let $x_n\in(\sqrt{2}, \sqrt{3})\cap \mathbb{Q}$ and $x_n\to a \in \mathbb{Q}$, then by the limit comparison theorem,
$$
\sqrt{2} \leq a \leq \sqrt{3}.
$$
Since $a\in\mathbb{Q}$, then
$$
\sqrt{2} < a < \sqrt{3}
$$
which implies $a\in(\sqrt{2}, \sqrt{3})\cap \mathbb{Q}$ and hence $(\sqrt{2}, \sqrt{3})\cap \mathbb{Q}$ is closed.

\item $\mathbf{(\sqrt{2}, \sqrt{3})\cap \mathbb{Q}\  is\ not\ compact}$

Let $\{x_n\}\in(\sqrt{2}, \sqrt{3})\cap \mathbb{Q}$ with $x_1<x_2<\cdots$. Set $x_0 := 0$. Let 
$$
r_k=\min\left\{\frac{x_{k+1}-x_k}{2}, \frac{x_k-x_{k-1}}{2}\right\}.
$$
At first, we know
$$
(\sqrt{2}, \sqrt{3})\cap \mathbb{Q} \subset \bigcup_{k=1}^{\infty} B_{r_k}(x_k).
$$

Secondly, we show that open balls are disjoint to each other. Suppose to contrary that they are not disjoint to each other, W.L.O.G, we assume $i<j$, then there is a point $x\in B_{r_i}(x_i)\cap B_{r_j}(x_j)$. Hence,
$$
\left| x-x_i \right| < r_i \text{ and } \left| x-x_j \right| < r_j.
$$
which implies
\begin{align*}
x &< x_i+r_i \leq x_i+\frac{x_{i-1}-x_i}{2} = \frac{x_i+x_{i-1}}{2}; \\
x &> x_j-r_j \geq x_j-\frac{x_j-x_{j-1}}{2} = \frac{x_j+x_{j-1}}{2}.
\end{align*}
Since the index $i<j$, and then
$$
x < \frac{x_i+x_{i-1}}{2} < \frac{x_j+x_{j-1}}{2} < x.
$$
which is a contradiction.

If $(\sqrt{2}, \sqrt{3})\cap \mathbb{Q}$ is compact, then there exists $N\in\mathbb{N}$ such that 
$$
(\sqrt{2}, \sqrt{3})\cap \mathbb{Q} \subseteq \bigcup_{k=1}^{N} B_{r_k}(x_k).
$$
But the open balls are disjoint, so there is a point $x\in(\sqrt{2}, \sqrt{3})\cap \mathbb{Q}$ such that
$$
x\notin\bigcup_{k=1}^{N} B_{r_k}(x_k).
$$
We conclude $(\sqrt{2}, \sqrt{3})\cap \mathbb{Q}$ is not compact.
\end{itemize}
\end{proof}

\item
\begin{solution}
Set $H_k := \left( \sqrt{2}-\frac{1}{k}, \sqrt{2}+\frac{1}{k} \right) \cap \mathbb{Q}$ for all $k\in\mathbb{N}$. A similar argument in part b) can get $H_k$ is closed and bounded for each $k\in\mathbb{N}$. Notice that
$$
H_1 \subset H_2 \subset \cdots.
$$
We know
$$
\bigcap_{k=1}^{\infty} H_k = \left\{\sqrt{2}\right\} \text{ and } \bigcap_{k=1}^{\infty} H_k \subseteq \mathbb{Q}.
$$
Since $\sqrt{2} \notin \mathbb{Q}$, this leads to a contradiction. Hence,
$$
\bigcap_{k=1}^{\infty} H_k = \phi
$$
might hold.
\end{solution}
\end{enumerate}
\end{Exercise}