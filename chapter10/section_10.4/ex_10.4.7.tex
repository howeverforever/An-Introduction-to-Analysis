% === Exercise 10.4.7 ===
\begin{Exercise}
\begin{proof}
We denote 
$$
f(x) 
:= {\rm dist}(x,B) 
= \left\{\rho(x,y):y\in B\right\}.
$$

Since $A,B$ are compact and $X$ has Bolzano-Weierstrass Property, then $A$ and $B$ are closed and bounded.
\begin{itemize}
\item $\mathbf{Claim\ \mathit{f}\ is\ continuous\ on\ \mathit{A}}$

Given $\epsilon>0$, for any $y\in B$, there exists $\delta = \epsilon > 0$ and $x_1, x_2\in A$ such that $\rho(x_1,x_2) < \delta$. Then 
\begin{alignat*}{7}
\quad&& \rho(x_1,y) &\leq \rho(x_1,x_2) + \rho(x_2, y)\quad&\text{and}\quad&& \rho(x_2,y) &\leq \rho(x_2,x_1) + \rho(x_1, y)  \\
\implies\quad&& f(x_1) &\leq \rho(x_1,x_2) + f(x_2)\quad&\text{and}\quad&& f(x_2) &\leq \rho(x_1,x_2) + f(x_1) \\
\implies\quad&& f(x_1)-f(x_2) &\leq \rho(x_1,x_2) \quad&\text{and}\quad&& f(x_2)-f(x_1) &\leq \rho(x_1,x_2)
\end{alignat*}
Hence,
$$
\left| f(x_1)-f(x_2) \right| \leq \rho(x_1, x_2) < \delta = \epsilon.
$$

It follows that $f$ is continuous on $A$.
\end{itemize}

Since $A$ is bounded and closed, and $f$ is continuous on $A$, by the Extreme Value Theorem, there exists $x_0\in A$ such that $f(x_0)=\inf_{x\in A}f(x)$.

Since $A\cap B = \emptyset$, for any $x\in A$ implies $x\notin B$. 

Since $B$ is closed, from exercise 10.3.5, we know $f(x) > 0$ for all $x\in A$. 

For any $x\in A$, we know
$$
\rho(x,y)
\geq f(x)
\geq f(x_0)
> 0
$$
for all $y\in B$. Hence $\rho(x,y) > 0$ for all $x\in A$ and $y\in B$. 

By definition of dist function, we conclude
$$
{\rm dist}(A,B) > 0.
$$

\begin{itemize}
\item $\mathbf{If\ A,B\ are\ only\ closed}$

Let
\begin{align*}
A &:= \left\{(x,1):x\geq 0\right\} \\
B &:= \left\{(x,y):y=\frac{x}{1+x},\ x\geq 0\right\}.
\end{align*}
i.e., $A,B$ are not bounded. We know $A\cap B=\emptyset$ as desired. However, set $(x, y) \in B$. $y\to 1$ as $x\to\infty$. Pick $(x, 1)\in A$ where $x$ is the same. In this view, dist$(A,B)=0$.
\end{itemize}
\end{proof}
\end{Exercise}