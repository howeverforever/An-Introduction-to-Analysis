% === Exercise 10.1.8 ===
\begin{Exercise}
\begin{enumerate}[a)]
\item
\begin{proof}
Given $\epsilon>0$, there exists $N\in\mathbb{N}$ such that $$
n,m\geq N \implies \sup_{x\in[a,b]}\left| f_n(x) - f_m(x) \right| < \epsilon.
$$
Taking $m\to\infty$ on both sides of the second inequality, we have
$$
\sup_{x\in[a,b]}\left| f_n(x) - f(x) \right| < \epsilon.
$$
Hence $f_n\to f$ uniformly converges on $[a,b]$.
Moreover, since $f_n\in C[a,b]$, we have $f\in C[a,b]$. We conclude the metric space $C[a,b]$ with supremum norm is complete.
\end{proof}

\item 
\begin{proof}
We change the notation $"dist(a,b)"$ as $"\rho(a,b)"$. Suppose $f,g,h \in C[a,b]$.
\item [$\mathbf{Positive\ Definite}$]
$$
\rho(f,g)
= \| f-g \|_1
= \int_{a}^{b} |f(x)-g(x)| dx
\geq \int_{a}^{b} 0\ dx
= 0.
$$
The equality holds if and only if $f(x)=g(x)$ for all $x\in[a,b]$.
\item [$\mathbf{Symmetric}$]
$$
\rho(f,g)
= \| f-g \|_1
= \int_{a}^{b} |f(x)-g(x)| dx
= \int_{a}^{b} |g(x)-f(x)| dx
= \| g-f \|_1
= \rho(g,f).
$$
\item [$\mathbf{Triangle\ Inequality}$]
\begin{align*}
\rho(f,g)
= \| f-g \|_1
&= \int_{a}^{b} |f(x)-g(x)| dx \\
&= \int_{a}^{b} |f(x)-h(x)+h(x)-g(x)| dx \\
&\leq \int_{a}^{b} |f(x)-h(x)| dx + \int_{a}^{b} |h(x)-g(x)| dx \\
&= \| f-h \|_1 + \| h-g \|_1 \\
&= \rho(f,h) + \rho(h,g).
\end{align*}
Hence the dist function makes $C[a,b]$ a metric space.
\end{proof}

\item
\begin{proof}
Assume $C[a,b]$ is complete in the metric $\rho$. Then $\{f_n\}$ converges to some functions which belongs to $C[a,b]$. 

Set $f_n = x^n$ and $[a,b] = [0,1]$ and
$$ 
g(x) = \begin{cases}0 & \mbox{ for } 0\leq x < 1 \\
1 & \mbox{ for } x = 1 \end{cases}.
$$
We know $\rho(f_n,g)\to 0$ as $n\to\infty$ which implies $g\in C[0,1]$. However $g$ is discontinuous at $x=1$. i.e., $g\notin C[0,1]$. This leads to a contradiction to $g\in C[0,1]$. 

Hence we conclude $C[a,b]$ is not complete.
\end{proof}
\end{enumerate}
\end{Exercise}