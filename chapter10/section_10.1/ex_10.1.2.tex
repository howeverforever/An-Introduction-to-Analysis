% === Exercise 10.1.2 ===
\begin{Exercise}
\begin{proof}
$(\Longrightarrow)$
There exists a number $M>0$ and $b\in X$ such that $\rho(x_k, b) \leq M$ for all $k\in\mathbb{N}$. Notice $\rho(b, a)$ is finite since $a, b\in X$. Hence
$$
\rho(x_k,a) 
\leq \rho(x_k,b) + \rho(b,a) 
\leq M + \rho(b,a) < \infty
$$ 
for all $k\in\mathbb{N}$ which implies $$
\sup_{k\in\mathbb{N}}\rho(x_k,a) 
< \infty.
$$

\vspace{2ex}

$(\Longleftarrow)$
Since 
$$
\sup_{k\in\mathbb{N}}\rho(x_k,a) < \infty,
$$ 
we choose $M>0$ such that $\rho(x_k,a) \leq M$ for all $k\in\mathbb{N}$. Then by definition, since $a\in X$, we conclude $\{x_k\} $ is bounded in $X$.
\end{proof}
\end{Exercise}