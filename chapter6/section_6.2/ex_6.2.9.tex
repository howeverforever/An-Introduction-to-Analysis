% === Exercise 6.2.9 ===
\begin{Exercise}
\begin{proof}
$(\Longrightarrow)$
Let $s_n = \sum_{k=1}^{\infty}a_k$. Since 
$$
\sum_{k=1}^{\infty}a_k 
= \lim_{n\to\infty}\sum_{k=1}^{n}a_k 
= \lim_{n\to\infty}s_n 
= S \in \mathbb{R}
$$
converges, so does its partial sum. Then 
$$
\sum_{k=1}^{\infty}(a_{2k}+a_{2k+1}) 
= \lim_{n\to\infty}(s_{2n+1}-a_1) 
= S-s_1 \in \mathbb{R}
$$ 
converges. 

\vspace{2ex}

$(\Longleftarrow)$ 
Since
$$
\sum_{k=1}^{\infty}(a_{2k}+a_{2k+1}) 
= \lim_{n\to\infty}(s_{2n+1}-a_1)
$$ 
converges so does $s_{2n+1}$. 

Consider 
$$
s_{2n+2} = s_{2n+1}+a_{2n+2}.
$$
Taking the limit of both sides as $n\to\infty$, we have 
$$
\lim_{n\to\infty}s_{2n+2}
= \lim_{n\to\infty}s_{2n+1}
$$
which implies $s_{2n+2}$ also converges.

Finally, no matter the number of terms is either odd or even, the series always converges; hence, we conclude 
$$ 
\sum_{k=1}^{\infty}a_k
$$ 
converges.
\end{proof}
\end{Exercise}