% === Exercise 6.1.9 ===
\begin{Exercise}
\begin{enumerate}[a)]
\item
\begin{proof}
For all $n>N$,
\begin{align*}
\left|n b-\sum_{k=1}^{n}b_k\right| 
&= \left|\sum_{k=1}^{n}(b-b_k)\right| \\
&\leq  \sum_{k=1}^{n}|b-b_k| \\
&\leq \sum_{k=1}^{N} |b-b_k| + \sum_{k=N+1}^{n}M \\
&= \sum_{k=1}^{N}|b_k-b| + M(n-N).
\end{align*}
\end{proof}

\item
\begin{proof}
Since $\lim_{n\to\infty} b_n = b$, given $\epsilon>0$, $\exists N\in\mathbb{N}$ such that $n \geq N \implies |b_n-b| < \epsilon$.
Then when $n > N$, by part a), we have 
$$
\left|\frac{b_1+b_2+\cdots+b_n}{n}-b\right| 
= \left|\frac{b_1+b_2+\cdots+b_n-n b}{n}\right| 
\leq \left|\frac{\sum_{k=1}^{n}\left|b_k-b\right|}{n}\right|+\epsilon\left(1-\frac{N}{n}\right).
$$

Taking the limit of both sides as $n\to\infty$, we have 
$$
\limsup_{n\to\infty}\left|\frac{b_1+b_2+\cdots+b_n}{n}-b\right| 
\leq \epsilon.
$$

This hold for any $\epsilon > 0$, so 
$$
\lim_{n\to\infty}\left|\frac{b_1+b_2+\cdots+b_n}{n}-b\right| = 0 
$$
which implies 
$$
\frac{b_1+b_2+\cdots+b_n}{n}\to b \text{ as } n\to\infty.
$$
\end{proof}

\item
\begin{solution}
The counter-example is $b_k=(-1)^{k-1}$.
\end{solution}
\end{enumerate}
\end{Exercise}