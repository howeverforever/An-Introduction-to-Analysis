% === Exercise 6.3.5 ===
\begin{Exercise}
\begin{proof}
Since $0<\frac{1}{k}\leq1 \mbox{ and } \sin{\frac{1}{k}} > 0$ for all $k\in\mathbb{N}$, we know 
$$ 
1+k\sin{\frac{1}{k}} > 0.
$$
Hence 
$$ 
\lim_{k\to\infty}\left|\frac{a_k}{a_{k-1}}\right| 
= \lim_{k\to\infty}\left(\frac{1}{1+k\sin{\frac{1}{k}}}\right) 
= \frac{1}{2} < 1 
$$ 
where 
$$ 
\lim_{k\to\infty}\left(k\sin{\frac{1}{k}}\right) 
= \lim_{k\to\infty}\left(\frac{\sin{\frac{1}{k}}}{\frac{1}{k}}\right) 
= \lim_{k\to\infty}\left(\cos{\frac{1}{k}}\right) 
= 1.
$$

By the Ratio Test, we conclude 
$$
\sum_{k=1}^{\infty}a_k
$$
converges absolutely.
\end{proof}
\end{Exercise}
